% latex_exam_template.tex (Version 2021.2)
%
% A LaTeX template for written exams.
% Wouter Grouve (https://personen.utwente.nl/w.j.b.grouve)
% Faculty of Engineering Technology, University of Twente.
%
% Based on the excellent template from the Australian National
% Unversity, created by Timothy Kam in 2004. The original version can
% be found here: https://ctan.org/pkg/anufinalexam?lang=en
%
% Licence type: Free as defined in the GNU General Public Licence:
% http://www.gnu.org/licenses/gpl.html

\documentclass[a4paper,12pt,fleqn]{article}
\usepackage{lastpage}
\usepackage{xcolor}
\usepackage{amsmath}
\usepackage{fancyhdr}
\usepackage{enumitem}
\usepackage{graphicx}
\usepackage{tabularx}
\usepackage[skip=2pt,font=small]{caption}
\usepackage{environ}
\usepackage{mdframed}

% Booleans to show answers and to ask students to return the question form
\newif\ifshowanswers
\newif\ifreturnform

% ==============================================================================
% Question command
%
\newcounter{question}
\newcommand*\question{%
  \stepcounter{question}%
  \paragraph{Question \thequestion}}

% ==============================================================================
% Answer boxes
%
\mdtheorem[outerlinewidth=2,roundcorner=10pt,
           leftmargin=0,rightmargin=0,
           backgroundcolor=yellow!40,outerlinecolor=blue!70!black,
           innertopmargin=\topskip,splittopskip=\topskip,
           ntheorem=true,]{answer_box}{Answer}[section]

\NewEnviron{answer}{
  \ifshowanswers
  \begin{answer_box*}
    \BODY
  \end{answer_box*}
  \fi}

% ==============================================================================
% Show answers?
%
\showanswerstrue
% \showanswersfalse

% ==============================================================================
% Return form?
%
% \returnformtrue
\returnformfalse


% ==============================================================================
%
% Course information
%
% ==============================================================================
\newcommand{\institution}{{\Large UNIVERSITY OF TWENTE}\\\vspace{4mm}
                                  Faculty of Engineering Technology\\
                                  Minor Aircraft Engineering}

\newcommand{\coursename}{Aircraft Structures}
\newcommand{\coursecode}{202000157}
\newcommand{\frontimage}{img/plane.png}

\newcommand{\examtype}{Final Exam}
\newcommand{\examdate}{January 2021}
\newcommand{\examtime}{Two hours}

\newcommand{\materials}{Calculator, Formula Sheet}
\newcommand{\lastwords}{End of Examination}


% ==============================================================================
%
% Margins, header and footer
%
% ==============================================================================
\setlength{\topmargin}{0cm}
\setlength{\textheight}{9.25in}
\setlength{\oddsidemargin}{0.0in}
\setlength{\evensidemargin}{0.0in}
\setlength{\textwidth}{16cm}
\pagestyle{fancy}
\cfoot{\footnotesize{Page \thepage \ of \pageref{finalpage}
       -- \coursename \ (\coursecode)}}
\renewcommand{\headrulewidth}{0pt}
\renewcommand{\footrulewidth}{0pt}

\begin{document}

% ==============================================================================
%
% Title page
%
% ==============================================================================
\thispagestyle{empty}

\begin{center}
\large\textbf{\institution}
\end{center}
\vspace{1cm}

\begin{center}
\vspace{1cm}
\includegraphics[scale=.8]{\frontimage}
\end{center}

\vspace{2cm}

\begin{center}
\Large\textbf{\coursename} (\coursecode)
\end{center}

\begin{center}
\textit{ \examtype{} -- \examdate}
\end{center}
\vspace{1cm}

\vspace{2cm}

\begin{center}
\textit{Available Time:  \examtime}
\end{center}
\begin{center}
  \textit{Permitted Materials: \materials}
\end{center}

\ifreturnform
\begin{table}[b]
  \centering
  \begin{tabular}{p{6cm}p{5cm}}\hline\\[-7pt]
    Name: & Student number:\\[5pt]\hline
  \end{tabular}
\end{table}
\fi

% ==============================================================================
%
% Second page: Generally used for instruction
%
% ==============================================================================

\newpage
\setcounter{page}{1}

\begin{quote}
  Generic guidelines:
    \begin{enumerate}
      \ifreturnform
      \item Write your name and student number on top of this page and
        hand this document at the end of the exam. Your exam will not be
        graded if you fail to return this question form.
      \fi
      \item The front page lists the materials you can use during this
        exam. Any other materials are not allowed. This includes mobile
        devices.
      \item When asked, elaborate your answers by providing the
        equations used or listing the assumptions made.
      \item Please write clearly. I am an engineer not an
        archaeologist.
    \end{enumerate}
    \vspace{3mm}
    \textbf{NB. This is an individual exam.} Good luck!
\end{quote}
\bigskip


% ==============================================================================
%
% Exam questions
%
% ==============================================================================

\newpage

\question [ 15/100 points ] The two dimensional stress state at a
particular point in a structure equals \mbox{$\sigma_{\text{x}} =$ 350
  MPa}, \mbox{$\sigma_{\text{y}} =$ 225 MPa} and
\mbox{$\tau_{\text{xy}} =$ 100 MPa}.
\begin{enumerate}
\item{} [ 10 points ] Calculate the principal stresses and the
  orientation of the corresponding principal planes.
  \begin{answer}
    The principal stresses are: $\sigma_{\textrm{I}} = 405$ MPa and
    $\sigma_{\textrm{II}} = 170$ MPa, while the principal planes are
    oriented at $\theta = 0.51 \pm \pi/2$ or
    $\theta = 29^{\circ} \pm 90^{\circ}$.

    As a reminder, the equations for the principal stresses are:
    \begin{equation*}
      \sigma_{\textrm{I,II}} = \frac{\sigma_{\textrm{x}} + \sigma_{\textrm{y}}}{2} \pm
      \frac{1}{2}\sqrt{(\sigma_{\textrm{x}} - \sigma_{\textrm{y}})^2 + 4\tau^2_{\textrm{xy}}},
    \end{equation*}
    while the principal planes can be found using:
    \begin{equation*}
      \tan 2\theta = \frac{2\tau_{\textrm{xy}}}{\sigma_{\textrm{x}} - \sigma_{\textrm{y}}}.
    \end{equation*}
    Points: Award 3 points for the correct equation for the principal
    stresses, 3 points for the correct equation for the principal
    planes and 2 points for each correct answer.
  \end{answer}
\item{} [ 5 points ] What is the magnitude of the shear stress
  acting on the principal planes?
  \begin{answer}
    The shear stress on the principal planes equals 0 MPa.

    Points: 5 points for the correct answer, otherwise 0.
  \end{answer}
\end{enumerate}

\question [ 15/100 points ] The turbine blades in modern jet engines are
subjected to intense heat and extreme loads\ldots

\begin{center}
\vspace{3cm}
--------- \textit{\lastwords} ---------
\end{center}

\label{finalpage}
\end{document}